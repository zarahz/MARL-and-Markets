% -------------------------------------------------------------------------------------------------
%      MDSG Latex Framework
%      ============================================================================================
%      File:                  abstract.tex
%      Author(s):             Michael Duerr
%      Version:               1
%      Creation Date:         30. Mai 2010
%      Creation Date:         30. Mai 2010
%
%      Notes:                 - Place your abstract here
% -------------------------------------------------------------------------------------------------
%
\vspace*{2cm}

\begin{center}
    \textbf{Abstract}
\end{center}

\vspace*{1cm}

\noindent A current research question addresses the topic of how to bring mixed-motive agents to work together, in order to achieve a common goal. Mixed-motive describes agents that work independently and whose actions do not affect others directly. However, the agents are not able to communicate. One approach is to introduce markets, namely a shareholder market (SM) and an action market (AM). By using markets, agents gain incentives, when they act cooperatively. Shares of the SM let agents participate in the reward of others and AM enable agents to reward others for certain actions.

This thesis introduces the coloring environment and uses it to compare the application of the two markets in various agent compositions. The coloring environment lets agents move around and color the cells they visit. Visiting a cell that is already colored removes its color. The goal is to color the whole environment.

The rewards of competitive and mixed-motive agents are calculated with the amount of color presence in the environment. Cooperative agents however get one shared reward based on the overall coloration, which can lead to the credit assignment problem (CAP). Additionally, all three compositions face organizational challenges of getting in each other's way. The effectiveness of markets on these problems are analyzed in this research. 