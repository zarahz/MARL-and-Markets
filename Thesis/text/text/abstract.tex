\vspace*{2cm}

\begin{center}
    \textbf{Abstract}
\end{center}

\vspace*{1cm}

\noindent A current research addresses the question of how to get mixed-motive agents to work together to achieve a common goal. Mixed-motive is an agent composition, that describes agents working independently and whose actions do not affect others directly. In most cases, the agents are not able to communicate. The influence to work in cooperation could be established through markets, namely a shareholder market (SM) or an action market (AM). By using markets, agents gain incentives when they act cooperatively. Shares of the SM let agents participate in the reward of others and an AM enables agents to reward others for certain actions.

This thesis introduces the coloring environment and uses it to compare the impact of the two markets in three different agent compositions - cooperation, mixed-motive and competitive. The coloring environment lets agents move around and color the cells they visit. Visiting a cell that is already colored removes its color, unless the competitive setting is set. In this case, the agent can capture opponent cells. The goal is to color the whole environment.

Furthermore, the compositions are established by means of reward distribution. Rewards of competitive and mixed-motive agents are calculated with their individual amount of color presence in the environment. Cooperative agents however, get one shared reward based on the overall coloration, which can lead to the credit assignment problem (CAP). Additionally, all three compositions face organizational challenges of agents getting in each other's way. The effectiveness of markets on these problems are analyzed in this research. 